% Options for packages loaded elsewhere
\PassOptionsToPackage{unicode}{hyperref}
\PassOptionsToPackage{hyphens}{url}
%
\documentclass[
]{article}
\usepackage{amsmath,amssymb}
\usepackage{lmodern}
\usepackage{iftex}
\ifPDFTeX
  \usepackage[T1]{fontenc}
  \usepackage[utf8]{inputenc}
  \usepackage{textcomp} % provide euro and other symbols
\else % if luatex or xetex
  \usepackage{unicode-math}
  \defaultfontfeatures{Scale=MatchLowercase}
  \defaultfontfeatures[\rmfamily]{Ligatures=TeX,Scale=1}
\fi
% Use upquote if available, for straight quotes in verbatim environments
\IfFileExists{upquote.sty}{\usepackage{upquote}}{}
\IfFileExists{microtype.sty}{% use microtype if available
  \usepackage[]{microtype}
  \UseMicrotypeSet[protrusion]{basicmath} % disable protrusion for tt fonts
}{}
\makeatletter
\@ifundefined{KOMAClassName}{% if non-KOMA class
  \IfFileExists{parskip.sty}{%
    \usepackage{parskip}
  }{% else
    \setlength{\parindent}{0pt}
    \setlength{\parskip}{6pt plus 2pt minus 1pt}}
}{% if KOMA class
  \KOMAoptions{parskip=half}}
\makeatother
\usepackage{xcolor}
\IfFileExists{xurl.sty}{\usepackage{xurl}}{} % add URL line breaks if available
\IfFileExists{bookmark.sty}{\usepackage{bookmark}}{\usepackage{hyperref}}
\hypersetup{
  hidelinks,
  pdfcreator={LaTeX via pandoc}}
\urlstyle{same} % disable monospaced font for URLs
\setlength{\emergencystretch}{3em} % prevent overfull lines
\providecommand{\tightlist}{%
  \setlength{\itemsep}{0pt}\setlength{\parskip}{0pt}}
\setcounter{secnumdepth}{-\maxdimen} % remove section numbering
\ifLuaTeX
  \usepackage{selnolig}  % disable illegal ligatures
\fi

\author{}
\date{}

\begin{document}

\{\rtf1\ansi\ansicpg1252\cocoartf2636
\cocoatextscaling0\cocoaplatform0{\fonttbl\f0\fswiss\fcharset0 ArialMT;\f1\fswiss\fcharset0 Arial-BoldMT;\f2\froman\fcharset0 Times-Roman;
\f3\froman\fcharset0 TimesNewRomanPSMT;\f4\froman\fcharset0 TimesNewRomanPS-BoldMT;\f5\fswiss\fcharset0 Helvetica-Bold;
\f6\fswiss\fcharset0 Helvetica;\f7\fswiss\fcharset0 Helvetica-BoldOblique;}
\{\colortbl;\red255\green255\blue255;\red191\green191\blue191;\red128\green128\blue128;\}
\{*\expandedcolortbl;;\csgray\c79525;\csgenericrgb\c50196\c50196\c50196;\}
\paperw11900\paperh16840\margl1440\margr1440\vieww11520\viewh8400\viewkind0
\deftab708

\itap1\trowd \taflags1 \trgaph108\trleft-108 \trwWidth11410\trftsWidth2
\trbrdrt\brdrnil \trbrdrl\brdrnil \trbrdrb\brdrs\brdrw45\brdrcf3
\trbrdrr\brdrnil  \clvertalt \clshdrawnil \clwWidth8116\clftsWidth3
\clbrdrt\brdrs\brdrw20\brdrcf2 \clbrdrl\brdrs\brdrw20\brdrcf2
\clbrdrb\brdrs\brdrw45\brdrcf3 \clbrdrr\brdrs\brdrw45\brdrcf3 \clpadl100
\clpadr100 \gaph\cellx4320
\clvertalt \clshdrawnil \clwWidth1398\clftsWidth3
\clbrdrt\brdrs\brdrw20\brdrcf2 \clbrdrl\brdrs\brdrw45\brdrcf3
\clbrdrb\brdrs\brdrw45\brdrcf3 \clbrdrr\brdrs\brdrw20\brdrcf2 \clpadl100
\clpadr100 \gaph\cellx8640
\pard\intbl\itap1\tqc\tx4252\tqr\tx8504\pardeftab708\ri182\qj\partightenfactor0

\f0\fs16 \cf0 M'e1ster Universitario en Formaci'f3n del Profesorado de
Educaci'f3n Secundaria Obligatoria y Bachillerato, Formaci'f3n
Profesional y Ense'f1anza de Idiomas\\
\f1\b Asignatura: Aprendizaje y Desarrollo de la Personalidad \f0\b0\\
\f2 \cell 
\pard\intbl\itap1\tqc\tx4252\tqr\tx8504\pardeftab708\ri182\qr\partightenfactor0

\f1\b \cf0 2021-22 \f3\b0 \cell \lastrow\row
\pard\pardeftab708\li142\ri182\sl276\slmult1\sa200\qj\partightenfactor0

\fs20 \cf0\\
\pard\pardeftab708\li510\ri182\partightenfactor0

\fs22 \cf0\\
\pard\pardeftab708\li510\ri182\partightenfactor0

\f4\b \cf0\\
\pard\pardeftab708\li510\ri182\partightenfactor0

\f5\fs28 \cf0 Parte OPCIONAL de la Pr'e1ctica 1\\
\pard\pardeftab708\li510\ri182\partightenfactor0

\f4\fs22 \cf0\\
\pard\pardeftab708\ri182\partightenfactor0

\f5 \cf0 \ul \ulc0 Sondeo. \f6\b0\\
\pard\pardeftab708\ri182\partightenfactor0

\f3 \cf0 \ulnone \\
\pard\pardeftab708\ri182\qj\partightenfactor0

\f5\b \cf0 Objetivo \f6\b0 : Conocer las \ul caracter'edsticas que puede
atribuir el \ulnone alumnado de secundaria /o bachillerato a los que
consideran \f7\i\b mejores y peores \f6\i0\b0 profesores.\\
\strut \\
En base a la disponibilidad de cada persona, se tratar'eda de realizar
una peque'f1a encuesta o sondeo de indagaci'f3n entre alumnado de
secundaria sobre ello, es decir sobre esas caracter'edsticas
atribu'edas. La idea es elaborar un peque'f1o an'e1lisis con los datos
que pueden obtener y ver si guardan relaci'f3n los aspectos analizados
en la actividad que realizamos en clase sobre ello. \f3\\
\pard\pardeftab708\ri182\qj\partightenfactor0 \cf0 \strike \strikec0\\
\pard\pardeftab708\ri182\qj\partightenfactor0

\f6 \cf0 \strike0\striked0 En caso de que no puedas acceder a esta
poblaci'f3n de estudiantes, podr'e1s elaborar la actividad tomando como
referencia tu paso por las aulas del instituto, universidad, o ambas.
\f3\\
\strut \\

\f6 Esta parte de la actividad pueden subirla al espacio tarea
\f5\b (parte opcional Practica 1 M'f3dulo II) \f6\b0 que encontrar'e1n
el aula virtual. \f3\\
\}

\end{document}
