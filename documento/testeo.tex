% Options for packages loaded elsewhere
\PassOptionsToPackage{unicode}{hyperref}
\PassOptionsToPackage{hyphens}{url}
%
\documentclass[
]{article}
\usepackage{amsmath,amssymb}
\usepackage{lmodern}
\usepackage{iftex}
\ifPDFTeX
  \usepackage[T1]{fontenc}
  \usepackage[utf8]{inputenc}
  \usepackage{textcomp} % provide euro and other symbols
\else % if luatex or xetex
  \usepackage{unicode-math}
  \defaultfontfeatures{Scale=MatchLowercase}
  \defaultfontfeatures[\rmfamily]{Ligatures=TeX,Scale=1}
\fi
% Use upquote if available, for straight quotes in verbatim environments
\IfFileExists{upquote.sty}{\usepackage{upquote}}{}
\IfFileExists{microtype.sty}{% use microtype if available
  \usepackage[]{microtype}
  \UseMicrotypeSet[protrusion]{basicmath} % disable protrusion for tt fonts
}{}
\makeatletter
\@ifundefined{KOMAClassName}{% if non-KOMA class
  \IfFileExists{parskip.sty}{%
    \usepackage{parskip}
  }{% else
    \setlength{\parindent}{0pt}
    \setlength{\parskip}{6pt plus 2pt minus 1pt}}
}{% if KOMA class
  \KOMAoptions{parskip=half}}
\makeatother
\usepackage{xcolor}
\IfFileExists{xurl.sty}{\usepackage{xurl}}{} % add URL line breaks if available
\IfFileExists{bookmark.sty}{\usepackage{bookmark}}{\usepackage{hyperref}}
\hypersetup{
  hidelinks,
  pdfcreator={LaTeX via pandoc}}
\urlstyle{same} % disable monospaced font for URLs
\usepackage{longtable,booktabs,array}
\usepackage{calc} % for calculating minipage widths
% Correct order of tables after \paragraph or \subparagraph
\usepackage{etoolbox}
\makeatletter
\patchcmd\longtable{\par}{\if@noskipsec\mbox{}\fi\par}{}{}
\makeatother
% Allow footnotes in longtable head/foot
\IfFileExists{footnotehyper.sty}{\usepackage{footnotehyper}}{\usepackage{footnote}}
\makesavenoteenv{longtable}
\setlength{\emergencystretch}{3em} % prevent overfull lines
\providecommand{\tightlist}{%
  \setlength{\itemsep}{0pt}\setlength{\parskip}{0pt}}
\setcounter{secnumdepth}{-\maxdimen} % remove section numbering
\ifLuaTeX
  \usepackage{selnolig}  % disable illegal ligatures
\fi

\author{}
\date{}

\begin{document}

\textbf{Metodología: La planificación de la investigación}

En este apartado vamos a abordar todas las tareas relacionadas con la
metodología que vamos a utilizar para desarrollar la investigación del
tema elegido.

\textbf{Decidir qué contenidos se van a \emph{investigar} y cuáles se
van a \emph{explicar} (Tarea grupal, en clase)}

Este proceso exige que el profesorado reflexione sobre las posibilidades
didácticas de la selección de contenidos que acaba de realizar. Supone
decidir los aprendizajes que considera que debe explicar (es decir,
enseñar mediante una estrategia expositiva porque no se pueden aprender
de otra forma; porque no disponemos de los recursos necesarios para su
investigación; porque no disponemos del tiempo necesario; porque implica
un gasto inasumible; \ldots), y los que el alumnado deberá investigar
(es decir, aprender mediante una estrategia por descubrimiento). No
obstante, esta decisión no puede ser arbitraria, también hay que tener
en cuenta la correspondencia entre tipos de contenidos y tipos de
estrategias de enseñanza, pues de lo contrario no se producirá un
aprendizaje significativo para el alumnado.

Para realizar esta tarea se retoman los conocimientos identificados
anteriormente por cada asignatura en relación con las preguntas del
proyecto:

\begin{longtable}[]{@{}c@{}}
\toprule
\textbf{Asignatura:} \\
\midrule
\endhead
\textbf{Categorías y preguntas seleccionadas} \\
 \\
 \\
\bottomrule
\end{longtable}

Y decidimos cuáles los vamos a investigar y cuáles los vamos a explicar.

Una vez adoptada esta decisión se procederá a la planificación de las
correspondientes investigaciones y explicaciones.

\textbf{Planificación de la \emph{investigación}}

Esta tarea es la que debería de realizar el alumnado en relación con sus
investigaciones \textbf{(Los grupos} \textbf{de organizan para
investigar el tema).} Aquí la vamos a realizar nosotros (profesorado)
para saber y practicar lo que deberían hacer nuestros alumnos.

Para planificar la investigación de cualquier tema o problema, se deben
tener en cuenta las siguientes cuestiones relativas al tema de
investigación elegido y a las cuestiones planteadas:

\textbackslash1. ¿Qué información debemos recoger?

Identificación y concreción (si procede) de la información que nos hace
falta recoger para que el alumnado lleve a cabo la investigación, en
función de las preguntas y el curriculum que hemos seleccionado.

\textbackslash2. ¿Cómo recoger y analizar esa información?

¿Qué situaciones, acciones, experimentos, etc., vamos a realizar para
que el alumnado pueda recoger la información que le permita construir
los conocimientos seleccionados y las respuestas a las preguntas
planeadas.

La primera cuestión (¿Qué información debemos recoger?) ya está resuelta
mediante la tarea de análisis que hemos realizado previamente:

\begin{longtable}[]{@{}c@{}}
\toprule
\textbf{Asignatura:} \\
\midrule
\endhead
\textbf{Categorías y preguntas seleccionadas} \\
 \\
 \\
\bottomrule
\end{longtable}

Por tanto, ahora nos vamos a centrar en la tarea ¿Cómo recoger esa
información?

\textbf{¿Cómo recoger esa información?}

Esta tarea implica diseñar las situaciones (acciones, experimentos,
etc.), que el alumnado deberá de realizar para recoger la información
que le permita construir los conocimientos seleccionados y las
respuestas a las preguntas planeadas.

En algunos casos, la información que el alumnado pueda encontrar en
distintas fuentes (libros, revistas, internet, \ldots), será suficiente
para responder a las preguntas planteadas, en otros casos (los más
deseables), tendrá que obtener la información necesaria para la
construcción del conocimiento implícito a las cuestiones planteadas para
poder responderlas.

Para el desarrollo de este proceso metodológico deberemos:

\begin{enumerate}
\def\labelenumi{\arabic{enumi}.}
\tightlist
\item
  Describir individualmente las situaciones de aprendizaje de cada
  asignatura.
\item
  Consensuar las situaciones de aprendizaje para cada asignatura
\item
  Incorporar las situaciones de aprendizaje a la tabla anterior (en la
  que sustituiremos la columna BLOQUE Y EPIGRAFE por la de SITUACIONES
  DE APRENDIZAJE.
\item
  Integrar todas las situaciones de todas las asignaturas en una única
  tabla.
\item
  Coordinar la actuación de todas las asignaturas.
\end{enumerate}

\textbf{1. Describir individualmente las situaciones de aprendizaje.}

Las situaciones de aprendizaje se refieren a cualquier actividad,
acción, experimento, observación directa o indirecta, etc., que
decidamos realizar para que el alumnado recoja la información que
necesita para construir los conocimientos que ya hemos identificado y
poder responder a las preguntas formuladas.

Para ahorrar espacio, a partir de ahora nos referiremos a esta tarea
como SITUACIONES DE APRENDIZAJE, que puede implicar su mera DENOMINACIÓN
o su DESCRIPCIÓN detallada, lo que habrá que especificar en cada caso.

Esta tarea se realiza primero individualmente y luego en el grupo que se
ha constituido en torno a cada asignatura.

Para realizar esta descripción, tendremos en cuenta los siguientes
aspectos:

\begin{enumerate}
\def\labelenumi{\arabic{enumi}.}
\tightlist
\item
  La denominación de la situación de aprendizaje
\item
  Contenidos que se quieren trabajar\footnote{Las tareas 2 y 3 a veces
    es mejor presentarlas por separado y otras en una sola.}
\item
  ¿Qué se quiere conseguir con esta situación de aprendizaje?
\item
  La secuencia didáctica\footnote{La secuencia didáctica es el
    desarrollo de la \textbf{estrategia de enseñanza} que decidamos
    adoptar. Esta secuencia puede abarcar una o más sesiones. Debe
    indicarse el nº de sesiones total.}. En la descripción de cada paso
  de esa secuencia, deberemos tener en cuenta:
\end{enumerate}

Descripción de la actividad del alumnado y/o del profesorado

El tipo de agrupamiento (alumno aislado; gran grupo clase; pequeño
grupo)

El tiempo previsto para su realización

Los recursos necesarios

\textbf{Ejemplo 1\footnote{La descripción de esta situación de
  aprendizaje la realizó el grupo prácticas de la asignatura Biología y
  Geología del IES Pitagorín durante el curso 2028-19.}:}

\textbackslash1. Denominación de la situación de aprendizaje:
\textbf{``La Temperatura de la Tierra''}

\textbackslash2. Contenidos que se quieren trabajar: \textbf{Evolución
de las temperaturas como indicador del cambio climático.}

\textbackslash3. ¿Qué se quiere conseguir con esta actividad o situación
de aprendizaje?

\textbf{Comparar los datos de temperaturas}

\textbf{Elaborar un climograma}

\textbf{Explicar la relación entre la evolución de las temperaturas y el
cambio climático}

Nº de sesiones: 3

Materiales: Datos del clima de localidades en Canarias

\textbackslash4. Secuencia didáctica:

1º El profesor/a explica a la clase qué es un climograma, cómo se
construye y cómo se interpreta (Una sesión; aprendizaje individual)

2º Cada alumno/a comparará los datos recogidos por el Instituto Nacional
de Estadística en Santa Cruz de Tenerife, Izaña (municipio de La
Orotava) y Gando (municipios de Ingenio y Telde) sobre las temperaturas
en distintos periodos\footnote{Para que el alumnado se dé cuenta de que
  el cambio climático es un proceso cercano y que nos afecta a todos, se
  han elegido localidades cercanas en las Islas Canarias de la que se
  disponen datos. Cuando se realice la distribución de los datos
  estadísticos entre el alumnado, se ha de tener en cuenta que se
  entregan datos repetidos a grupos que están alejados en la clase para
  evitar que se transmitan la información, es decir, no poner a dos
  grupos con los mismos datos juntos. El alumnado ha de observar el
  cambio producido en las temperaturas a lo largo del tiempo y responder
  a las preguntas formuladas} (tarea individual; los materiales los
proporciona el profesorado):

\begin{itemize}
\tightlist
\item
  \textbf{1979}
  \url{http://www.ine.es/inebaseweb/pdfDispacher.do;jsessionid=59086F702022A87F6DCAFBFC0B9F7A82.inebaseweb01?td=123972}
\item
  \textbf{1988}
  h\href{http://www.ine.es/inebaseweb/pdfDispacher.do;jsessionid=59086F702022A87F6DCAFBFC0B9F7A82.inebaseweb01?td=134513}{ttp://www.ine.es/inebaseweb/pdfDispacher.do;jsessionid=59086F702022A87F6DCAFBFC0B9F7A82.inebaseweb01?td=134513}
\item
  \textbf{1996}
  \url{http://www.ine.es/inebaseweb/pdfDispacher.do?td=145952}
\item
  \textbf{2010}
  \url{http://www.ine.es/jaxi/Datos.htm?path=/t38/p604/a2000/l0/\&file=1900001.px}
\item
  \textbf{2015}
  \url{http://www.ine.es/jaxi/Datos.htm?path=/t38/p604/a2000/l0/\&file=1900001.px}
\end{itemize}

Como consecuencia de esa comparación elaborará un texto en el que
explique la evolución del clima. Para guiar su reflexión y para ayudarle
a estructurar el texto, deberá intentar responder a las siguientes
preguntas:

\begin{itemize}
\tightlist
\item
  \emph{¿Cómo han cambiado las temperaturas?}
\item
  \emph{¿Qué diferencias hay entre las temperaturas de 1979, 1988, 1996,
  2010 y 2015?}
\item
  \emph{¿Son más altas o más bajas las temperaturas?}
\item
  \emph{¿Cómo creemos que van a ser las temperaturas en los próximos
  años?}
\end{itemize}

3º. Cada alumno presenta a su grupo sus reflexiones. Una vez presentadas
todas las reflexiones, el grupo debate y elabora un texto en el que
refleje sus ideas respecto a cada una de las preguntas formuladas (5
minutos cada uno = 20 minutos en total; 30 minutos de debate; una
sesión)

4º Cada grupo elige a un portavoz que presentará al resto de los grupo
sus conclusiones, explicando los datos que han recibido y como los han
interpretado (Cada grupo dispondrá de 7 minutos para exponer su trabajo;
una sesión)

\textbf{Ejemplo 2:}

\textbackslash1. Denominación de la situación de aprendizaje:
\textbf{Los materiales que desechamos}

\textbackslash2. Contenidos que se quieren trabajar/ ¿Qué se quiere
conseguir con esta actividad o situación de aprendizaje?

\textbf{Clasificación de objetos y materiales a partir de criterios
elementales físicas observables: olor textura, color, peso/masa, dureza,
estado físico o capacidad de disolución en agua.}

\textbf{Reconocimiento, en los objetos o cuerpos, de la propiedad de
longitud, peso/masa y capacidad, y comprensión del concepto de medida.}

\textbackslash4. Secuencia didáctica:

1º Cada alumno deberá de haber recogido diferentes tipos de residuos
(papel, cartón, plástico, restos orgánicos,\ldots). Como máximo cada
alumno aportará tres materiales de diferente naturaleza.

2º Cada alumno, en su casa, realiza las siguientes observaciones de cada
uno de los materiales recogidos: longitud, peso, textura, plasticidad,
estado de agregación, \ldots{} y las escribe en esta tabla:

\begin{longtable}[]{@{}llllll@{}}
\toprule
\textbf{Material} & \textbf{Longitud} & \textbf{Peso} & \textbf{Textura}
& \textbf{Plasticidad} & \textbf{Estado de agregación} \\
\midrule
\endhead
Caja de cartón & & & & & \\
Botella de vidrio & & & & & \\
Brick de leche & & & & & \\
\ldots. & & & & & \\
\bottomrule
\end{longtable}

3º El alumnado se reúne con su grupo y cada uno presenta las
observaciones anteriores. El grupo analiza y valora si esas
observaciones son o no correctas. Se corrigen los posibles errores
cometidos. (20 minutos; lápiz y papel).

4º) Cada grupo elaborará un informe de todos los materiales observados
mediante una tabla de doble entrada (materiales y propiedades. Además
los clasificará según las propiedades que decida el profesor (por
ejemplo: por la plasticidad; por la textura; por su estado de
agregación; etc.) (30 minutos, lápiz y papel):

\begin{longtable}[]{@{}llllll@{}}
\toprule
\textbf{Material} & \textbf{Longitud} & \textbf{Peso} & \textbf{Textura}
& \textbf{Plasticidad} & \textbf{Estado de agregación} \\
\midrule
\endhead
Caja de cartón & & & & & \\
Botella de vidrio & & & & & \\
Brick de leche & & & & & \\
\ldots. & & & & & \\
\bottomrule
\end{longtable}

5º) Cada grupo expondrá al resto de la clase sus observaciones y
conclusiones, con el consiguiente debate acerca de las observaciones
realizadas (20 minutos).

Número de sesiones: 2

\textbf{2. Consensuar las situaciones de aprendizaje para cada
asignatura}

Una vez que cada uno ha descrito las situaciones de aprendizaje que cree
más adecuadas para que el alumnado construya los conocimientos
necesarios para poder responder a las preguntas planteadas, procederemos
a consensuarlas para cada una de las asignaturas. De tal modo, que como
consecuencia de ese proceso cada asignatura dispondrá de un amplio
conjunto de situaciones de aprendizaje. Esta descripción seguirá las
mismas pautas que las individuales.

En este momento, y en situaciones reales, se pueden adoptar dos
decisiones diferentes:

\begin{itemize}
\tightlist
\item
  \textbf{Consensuamos las situaciones que nos parezcan más adecuadas y
  todos llevaremos a la práctica esas mismas situaciones}.
\item
  Asumimos que para los mismos conocimientos pude haber distintas
  situaciones adecuadas y cada profesor/a puede decidir las que llevará
  a la práctica.
\end{itemize}

Nosotros en esta práctica vamos a adoptar la primera decisión por
razones que tienen que ver más con el desarrollo de las prácticas de
esta asignatura que con lo que puede ocurrir en la práctica real.

**

\textbf{3. Incorporar las situaciones consensuadas a la nueva tabla y
sus correspondientes descripciones.}

Una vez descritas y consensuadas las situaciones de aprendizaje las
incorporamos a una nueva tabla para tener una visión global de lo que va
a trabajar cada asignatura.

\begin{longtable}[]{@{}c@{}}
\toprule
\textbf{ASIGNATURA} \\
\midrule
\endhead
\textbf{Categorías y preguntas seleccionadas} \\
 \\
 \\
\bottomrule
\end{longtable}

A continuación de esta tabla se incorporan las descripciones detalladas
de cada una de las situaciones consensuadas e incluidas en la tabla
anterior.

Por ejemplo:

\begin{longtable}[]{@{}c@{}}
\toprule
\textbf{Biología y Geología} \\
\midrule
\endhead
\textbf{Categorías y preguntas seleccionadas} \\
\textbf{I. CONCEPTOS BÁSICOS} \\
 \\
\textbf{II. POLÍTICA} \\
 \\
\textbf{III. REPERCUSIONES} \\
 \\
\textbf{IV. SOCIEDAD} \\
 \\
\textbf{V. POSIBLES SOLUCIONES} \\
 \\
\bottomrule
\end{longtable}

\textbf{Situación de aprendizaje 1: ``Cambio climático: Antes de que sea
tarde''\footnote{Las descripciones de las situaciones de aprendizaje las
  realizó el alumnado de la asignatura Biología y Geología del IES
  Pitagorín para las prácticas de esta asignatura durante el curso
  2018\_19.}}

\textbf{Nº de sesiones: 4}

\textbf{Materiales: Televisor o proyector, ordenador.}

Para comenzar a tratar el cambio climático y hacer que los alumnos
tengan una idea del cambio climático previa, se visualizará un
documental de Leonardo Dicaprio, titulado ``Antes de que sea
tarde'':\href{https://www.youtube.com/watch?v=LbRUSffD6pY}{}\url{https://www.youtube.com/watch?v=LbRUSffD6pY}

\href{https://www.youtube.com/watch?v=LbRUSffD6pY}{}Una vez se haya
visto el documental, se pasará un cuestionario previo a los alumnos para
ver las ideas que han obtenido por la visualización. El cuestionario
constará de las siguientes preguntas:

\begin{itemize}
\tightlist
\item
  \emph{¿Crees que el cambio climático es un problema real? ¿Por qué?}
\item
  \emph{¿Cuáles son las causas principales del problema?}
\item
  \emph{¿Qué efectos crees que se están viendo ya en La Tierra debidos
  al cambio climático?}
\item
  \emph{¿Qué problemas crees que tendremos en el futuro por el cambio
  climático?}
\item
  \emph{Escribe 5 medidas que pueden frenar el cambio climático}
\end{itemize}

Una vez el alumno se haya planteado todas estas cuestiones sobre el
tema, dividiremos la clase en dos mitades y se hará a los alumnos
debatir acerca del cambio climático según lo visto en el documental. Se
separará a los alumnos al azar e intentaremos generar un debate donde se
tomarán las siguientes posturas:

\begin{itemize}
\tightlist
\item
  \emph{Cambio climático problema real o ficticio. (Un grupo cada
  postura)}
\item
  \emph{Energías renovables o de combustibles fósiles. (Un grupo cada
  postura al revés que el debate anterior)}
\end{itemize}

Para finalizar, se pide a los alumnos que con lo visto en documental y
apoyándose en internet, respondan a las preguntas relacionadas con el
cambio climático planteadas, con criterio y de forma elaborada.

\textbf{Situación de aprendizaje 2: ``La Temperatura de la Tierra''}

\textbf{Nº de sesiones: 2}

\textbf{Materiales: Datos del clima de localidades en Canarias,
cuestionario con las preguntas a resolver y DIN-A4}

En primer lugar, en grupos de 4 o 5 personas tendrán que explicar la
evolución del clima de la Tierra el alumnado tendrá que comparar los
datos recogidos por el Instituto Nacional de Estadística en Santa Cruz
de Tenerife, Izaña (municipio de La Orotava) y Gando (municipios de
Ingenio y Telde) sobre las temperaturas en distintos periodos:

\begin{itemize}
\tightlist
\item
  \textbf{1979}
  \url{http://www.ine.es/inebaseweb/pdfDispacher.do;jsessionid=59086F702022A87F6DCAFBFC0B9F7A82.inebaseweb01?td=123972}
\item
  \textbf{1988}
  h\href{http://www.ine.es/inebaseweb/pdfDispacher.do;jsessionid=59086F702022A87F6DCAFBFC0B9F7A82.inebaseweb01?td=134513}{ttp://www.ine.es/inebaseweb/pdfDispacher.do;jsessionid=59086F702022A87F6DCAFBFC0B9F7A82.inebaseweb01?td=134513}
\item
  \textbf{1996}
  \url{http://www.ine.es/inebaseweb/pdfDispacher.do?td=145952}
\item
  \textbf{2010}
  \url{http://www.ine.es/jaxi/Datos.htm?path=/t38/p604/a2000/l0/\&file=1900001.px}
\item
  \textbf{2015}
  \url{http://www.ine.es/jaxi/Datos.htm?path=/t38/p604/a2000/l0/\&file=1900001.px}
\end{itemize}

Para transmitir al alumnado que es un proceso que no es lejano y que nos
afecta a todos, se han elegido localidades cercanas en las Islas
Canarias de la que se disponen datos. Cuando se realice la distribución
de los datos estadísticos entre el alumnado, se ha de tener en cuenta
que se entregan datos repetidos a grupos que están alejados en la clase
para evitar que se transmitan la información, es decir, no poner a dos
grupos con los mismos datos juntos. El alumnado ha de observar el cambio
producido en las temperaturas a lo largo del tiempo y responder a la
preguntas:

\begin{itemize}
\tightlist
\item
  \emph{¿Cómo han cambiado las temperaturas?}
\item
  \emph{¿Que diferencias hay entre las temperaturas de 1979, 1988, 1996,
  2010 y 2015?}
\item
  \emph{¿Son más altas o más bajas las temperaturas?}
\item
  \emph{¿Cómo creemos que van a ser las temperaturas en los próximos
  años?}
\end{itemize}

Se les dará al alumnado un tiempo de 35 minutos para completar la
actividad.

Después se explicará al alumnado cómo realizar un climograma durante 15
minutos. El alumnado deberá realizar por grupos de 4 o 5 personas
durante 40 minutos para posteriormente realizar una presentación oral al
resto de la clase sobre las conclusiones a las que llegaron explicando
los datos que han recibido y como los han interpretado. Cada grupo
dispondrá de 3 minutos para exponer su trabajo 20 minutos en total.

\textbf{Situación de aprendizaje 3: ``Los ecosistemas y el cambio
climático''}

\textbf{Nº de sesiones: 2}

\textbf{Materiales: Fichas con listados de seres vivos y listado de
animales en peligro de extinción:}

\url{https://es.wikipedia.org/wiki/Cat\%C3\%A1logo_de_Especies_Amenazadas_de_Canarias}

\textbf{Tablets o smartphones. Los alumnos han de disponer de material
de escritura y libreta o cuaderno.}

Se realizará, en primer lugar, una explicación teórica sobre las
relaciones inter e intraespecíficas en los ecosistemas a modo de
introducción y para poner al alumnado en situación:

\emph{Las \textbf{relaciones interespecíficas} en los ecosistemas son
aquellas que se establecen entre los seres vivos: ¿quién se come a
quién?; ¿cómo los animales se ayudan entre sí mediante relaciones de
simbiosis, por ejemplo: el pájaro dentista(pluvianidae) que limpia los
dientes del cocodrilo sin que este se lo coma?; ¿quién transporta a
quién?; ¿cuáles son los animales que viven en otros animales?; y los
parásitos. Las relaciones intraespecíficas son aquellas que se producen
dentro de la misma especie.}

Luego se procederá a explicar la importancia de los recursos naturales y
su escasez durante 20 minutos.

A partir de esta exposición, se presentará a cada uno de los grupos ya
formados de 4 o 5 personas en la clase una lista de seres vivos que
viven en un ecosistema canario diferente para cada grupo con el objetivo
de buscar las relaciones entre los seres vivos de dicho ecosistema a
través de la pregunta:

\begin{itemize}
\tightlist
\item
  \emph{¿Cómo se relacionan los seres vivos en este ecosistema?}
\end{itemize}

Los listados de seres vivos son los siguientes:

\emph{\textbf{El ecosistema tabaibal-cardonal:} caracoles terrestres, el
bisbita caminero, los mosquiteros, los herrerillos, los cernícalos, los
canarios, los conejos, los roedores, las musarañas, el balancón, la
uvilla de mar, los tarajales, el pincho, y la lecheruela.}

\emph{\textbf{El ecosistema de laurisilva:} el laurel, el naranjero
salvaje, el tilo, el sanguino, la tabaiba de monte, el paloblanco, el
viñatigo, el sacatero, el mocan, el sauce canario, el madroño canario,
el brezo, la orquídea de tres dedos, la col de risco, las chicharras
endémicas, la cucaracha, las tijeretas endémicas, el gorgojo, el
escarabajo apestoso, el gavilán, el aguililla, el cernícalo, la lechuza,
el búho chico, la paloma turquesa, el petirrojo, el vencejo, el canario
y las currucas.}

\emph{\textbf{El ecosistema de litoral canario:} la garza real, la
garceta común, los zarapitos, los correlimos, los chorlitejos, el
perejil de mal, la hubara canaria, la lechuga de mar, el salado, las
siemprevivas, las magarzas, el cangrejo ermitaño, la langosta, la
fabiana y la vaca de mar.}

\emph{\textbf{El ecosistema de bosques termófilos canarios:} el
acebuche, el lentisco, el mocan, los tajinastes, las jaras, la
vinagreras, la curruca, la cabecinegra, el aguilucho ratonero, el búho
chico, los saltamontes, la araña de las piteras, el robusto escarabajo
rinoceronte, la cochinilla, la chuchanga, el bisbita caminero, la
abubilla, la perdiz moruna, el triguero y la paloma bravía.}

\emph{\textbf{El ecosistema del pinar canario:} la tea, la brea, el pino
canario, los brezos, las jaras, el tomillo, el poleo, los gorgojos del
pino, la mariposa nocturna del pino, el pinzón azul, el muflón, el
escobón y el picapinos.}

\emph{\textbf{El ecosistema de alta montaña:} las retamas, los codesos,
la magarza del Teide, el alhelí del Teide, la violeta del Teide, el
lagarto tizón, el conejo, el erizo moruno, gran gorgojo del Teide,
escarabajos, la mariposa manto, cardo blanco, el rosal de la cumbre, la
salamanquesa, el nóctulo pequeño, la rata de campo, el gato cimarrón y
el bisbita camionero.}

El alumnado dispondrá de 45 minutos para completar la actividad. Al
terminar la actividad, los grupos tendrán que presentar a la clase los
ecosistemas que han trabajado durante 3 minutos. El resto de la clase
también podrá proponer nuevas relaciones entre seres vivos al grupo, si
lo estimase oportuno.

Las exposiciones en total tendrán una duración de 20 minutos. Luego se
les aportaran datos al alumnado sobre las distintas especies en peligro
de extinción. Los grupos deberán valorar que animales y plantas se verán
afectados por la desaparición de las especies en peligro de extinción
entre otras causas por el efecto del cambio climático. (20 minutos)
Deberán exponer el trabajo al resto de la clase durante 3 minutos
expresando las conclusiones a las que hayan llegado (20 minutos).

\textbf{Situación de aprendizaje 4: ``Síntomas de lo que se avecina''}

\textbf{Nº de sesiones:} 1

\textbf{Materiales: tablet u ordenador, a ser posibles los suficientes
para que toquen a uno por pareja o uno para cada uno en el mejor de los
casos.}

Seguiremos ahondando conocimientos climáticos, pero centrándonos cada
vez en las posibles consecuencias actuales y recientes del cambio
climático, enfocándonos en conceptos como desastres naturales o los
riesgos meteorológicos. De manera individual cada alumno responderá a
los siguientes ejercicios.

Lectura de pequeño texto y cuestionario:

\emph{¿El mundo se está quejando?}

\emph{Muchas personas dicen que ya no nieva tanto como antes, o que cada
vez los veranos son más largos y calurosos. Olas de calor, lluvias
torrenciales y huracanes cada vez más fuertes y más frecuentes son
fenómenos climáticos que han causado varias catástrofes en los últimos
años. Los científicos aseguran que el clima se está volviendo cada vez
más extremo.}

\emph{Las consecuencias son desastres naturales como sequías, incendios
forestales o inundaciones. Los ejemplos son muy claros. En verano de
2003 se vivió una fortísima ola de calor en Europa que todavía se
recuerda. El año 2004 se produjeron también en Europa 15 grandes
inundaciones que afectaron a un millón de personas. El año 2005 hubo
tantos huracanes que agotaron los nombres de la lista que los
científicos tienen preparada.} *

\textbf{\emph{Cuestionario:}}

\begin{itemize}
\tightlist
\item
  \emph{¿Qué fenómenos atmosféricos menciona el texto?}
\item
  \emph{¿Qué consecuencias tienen esos fenómenos atmosféricos?}
\item
  \emph{¿Guardan relación con el cambio climático? ¿Por qué?}
\item
  \emph{¿Sabes el significado de las siguientes palabras? Relaciónalo
  con las posibles respuestas:}
\end{itemize}

\texttt{}\emph{huracán- lluvia torrencial- sequía- tsunami- incendio
forestal- ola de calor- permafrost}

\begin{enumerate}
\def\labelenumi{\arabic{enumi}.}
\tightlist
\item
  \emph{Sistema tormentoso caracterizado por una circulación cerrada
  alrededor de un centro de baja presión que produce fuertes vientos y
  abundante lluvia.}
\item
  \emph{Precipitación con intensidad igual o superior a 1l/min.}
\item
  \emph{Periodo de tiempo más o menos prolongado (meses, años) donde las
  precipitaciones o son nulas o están muy por debajo de lo normal.}
\item
  \emph{Ola de gran tamaño y energía capaz de penetrar decenas o
  centenares de metros tierra adentro.}
\item
  \emph{Proceso mediante el cual una masa boscosa o de matorral arde.}
\item
  \emph{Situación meteorológica en la que por un determinado tiempo
  (días o semanas) se suceden temperaturas muy superiores a las normales
  para ese lugar.}
\item
  \emph{Capa del subsuelo en las regiones circumpolares que permanece
  helada durante todo el año.}
\end{enumerate}

Durante 25 min se resolverán estas preguntas de manera individual y
serán entregadas al profesor. Lo que restará de clase consistirá la
utilización de la herramienta de inundación del terreno de Grafcan:

https://visor.grafcan.es/visorweb/

Después de manejar la aplicación los alumnos realizarán de otro
cuestionario de manera grupal (5 alumnos) con las siguientes preguntas.
Para esta segunda mitad de sesión se requerirá de ordenador o Tablet.

\emph{Teniendo en cuenta los escenarios de subida de temperatura para el
año 2100, responde a las preguntas:}

\texttt{}\textbf{\emph{º C metros}}

\texttt{}\emph{1.8 0.30 -- 0.43}

\texttt{}\emph{2.4 0.51 -- 0.64}

\texttt{}\emph{3.4 0.78 -- 0.89}

\texttt{}\emph{4.0 1.10 -- 1.29}

\begin{itemize}
\tightlist
\item
  \emph{¿Qué cambios del territorio te llaman más la atención?}
\item
  \emph{¿El sitio en el que vives se vería afectado por la subida del
  nivel del mar?}
\item
  \emph{¿Qué partes de Canarias se verían más afectadas y cuáles menos?}
\item
  \emph{En el escenario de subida de 4.0 ºC, ¿quedaría alguna playa
  emergida en Tenerife?}
\end{itemize}

\textbf{Situación de aprendizaje 5: ``Cambio climático: Cada gesto
cuenta''}

\textbf{Nº de sesiones: 3}

\textbf{Materiales: Tablet o móvil, ordenador y proyector.}

Para la realización de esta actividad, separaremos los alumnos en grupos
de 5. Se le pondrá a los alumnos unos modelos de vídeos de ejemplo para
contribuir a frenar el cambio climático:

\begin{itemize}
\tightlist
\item
  \url{https://www.youtube.com/watch?v=U3DJuS9iY-wb}
\end{itemize}

\url{https://www.youtube.com/watch?v=BCvCVQmtSBI}A partir de estos
vídeos, los alumnos tendrán que realizar uno propio donde salgan
reflejado:

\begin{itemize}
\tightlist
\item
  Concienciación a la sociedad en el que deberán explicar la gravedad
  del problema,
\item
  Las posibles acciones individuales que se deben hacer para frenar el
  cambio climático\\
\item
  Cosas que no hace y podría hacer el instituto para mejorar en este
  tema.
\end{itemize}

Para ello, deberán recabar información de forma individual en la red
para enterarse de acciones que se pueden llevar a cabo para frenar el
cambio climático desde el punto de vista doméstico y aplicarlo también
al centro escolar. Una vez cada alumno haya conseguido 5 medidas para
evitar el cambio climático, se unirán en grupos y los pondrán en común
para la realización del vídeo. Esta parte de la actividad se llevará a
cabo en dos sesiones.

Por último, en una última sesión, se expondrán los videos de cada grupo
para que los compañeros vean, opinen y voten cada vídeo.

El mejor de los videos, será compartido por la página del IES y en las
redes sociales y será proyectado en las aulas del instituto dando
visibilidad al problema y concienciando a las personas que lo vean.

\textbf{Situación de aprendizaje 6: ``Los murales''}

\textbf{Nº de sesiones: 3}

\textbf{Materiales: Tablet, móvil u ordenador con conexión a internet,
materiales reciclados, material de escritura, material de dibujo y el
libro de texto de la asignatura.}

En alumnado en organizado en grupos de 4 o 5 personas tendrán que elegir
entre 7 temas para realizar un mural. Los temas son:

\begin{itemize}
\tightlist
\item
  Las iniciativas de reciclaje
\item
  La separación de residuos
\item
  La evolución del clima en la tierra
\item
  La contaminación
\item
  Las energías renovables
\item
  Los ecosistemas
\item
  Las iniciativas contra el cambio climático
\end{itemize}

Los grupos deberán organizarse para buscar información acerca del tema
que han elegido ya sea en Internet, en libros de texto, en la
información que han recibido en los días anteriores para preparar un
mural informativo que puede ser 2D o 3D.

Luego, deberán construir un mural en el que representen lo que creen,
piensan y saben del tema que han elegido. Este mural será elaborado con
materiales reciclables tales como revistas, periódicos o lo que ellos
dentro de los límites de la higiene y la salubridad consideren. (2 horas
20 minutos).

Posteriormente, los grupos presentarán sus murales al resto de la clase.
Cada grupo dispondrá de 10 minutos para explicar su mural y por qué han
decidido reflejar así el tema que les ha tocado. Han de informar también
al resto de los compañeros acerca del tema que defienden en su mural y
por qué es importante para todas las personas los efectos del cambio
climático. El alumnado podrán comentar a los compañeros que añadirían a
los murales presentados durante la clase. Además, el profesorado podrá
cuestionar al alumnado para ayudarles a entender los conocimientos
sobre:

\begin{itemize}
\tightlist
\item
  \emph{¿Por qué ha puesto determinados objetos, imágenes u otras
  representaciones?}
\item
  \emph{¿Qué significa para ellos haber colocado eso en el mural?}
\end{itemize}

Posteriormente, estos serán colgados o colocados en algún lugar de la
clase o del centro educativo.

\textbf{3. Integración de todas las situaciones de todas las asignaturas
en una única tabla, con sus correspondientes descripciones.}

Una vez que las asignaturas han consensuado las situaciones que van a
desarrollar, las integramos en una única tabla para que todo el
profesorado disponga de una visión global del conjunto de situaciones
que se van a desarrollar. Esa visión nos debería permitir mejorar la
coordinación entre el profesorado para evitar redundancias, establecer
la mejor secuencia posible de las situaciones, saber lo que el alumnado
ya ha aprendido cuando cada asignatura comienza a desarrollar sus
situaciones, etc.

\texttt{}Un ejemplo\footnote{Este ejemplo es la práctica que el curso
  2018-19 desarrolló el grupo denominado IES Pitagorín.}:

\begin{longtable}[]{@{}rccccc@{}}
\toprule
& & \textbf{FÍSICA Y QUÍMICA} & & & \\
\midrule
\endhead
\textbf{CATEGORÍA I: CONCEPTOS BÁSICOS} & & & & & \\
& & & & & \\
\textbf{CATEGORIA II: POLÍTICA} & & & & & \\
& & & & & \\
\textbf{CATEGORIA III: SOCIEDAD} & & & & & \\
& & & & & \\
\textbf{CATEGORÍA IV: REPERCUSIONES} & & & & & \\
& & & & & \\
\textbf{CATEGORÍA V: POSIBLES SOLUCIONES} & & & & & \\
& & & & & \\
\bottomrule
\end{longtable}

\textbf{DESCRIPCIÓN DETALLADA DE LAS SITUACIONES DE APRENDIZAJE:}

A continuación se incluirían las descripciones detalladas de cada una de
las situaciones de aprendizaje incluidas en la tabla anterior.

Los miembros de los grupos constituidos en torno a cada asignatura,
analizan la tabla y descripciones anteriores y elaboran sus necesidades
relativas a secuenciación, integración, etc. Una vez que cada asignatura
ha decidido sus necesidades, se procedería a coordinarse con las demás y
tomar las decisiones correspondientes.

Por ejemplo, en función del atabla anterior, se señala en rojo esas
necesidades:

\begin{longtable}[]{@{}rccccc@{}}
\toprule
& & \textbf{FÍSICA Y QUÍMICA} & & & \\
\midrule
\endhead
\textbf{CATEGORÍA I: CONCEPTOS BÁSICOS} & & & & & \\
& & & & & \\
\textbf{CATEGORIA II: POLÍTICA} & & & & & \\
& & & & & \\
\textbf{CATEGORIA III: SOCIEDAD} & & & & & \\
& & & & & \\
\textbf{CATEGORÍA IV: REPERCUSIONES} & & & & & \\
& & & & & \\
\textbf{CATEGORÍA V: POSIBLES SOLUCIONES} & & & & & \\
& & & & & \\
\bottomrule
\end{longtable}

\begin{longtable}[]{@{}l@{}}
\toprule
\endhead
\bottomrule
\end{longtable}

\end{document}
